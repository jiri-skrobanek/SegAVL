\documentclass[a4paper,10pt,openany]{article}
\usepackage[utf8]{inputenc}
%\usepackage[czech]{babel}
\usepackage{geometry}
\usepackage{tikz,multirow,listingsutf8,multicol,amsmath,amsfonts}
\lstset{language=haskell,showstringspaces=false,stringstyle=\color{orange},basicstyle=\ttfamily\small,keywordstyle=\color{blue!60!black}, morekeywords={Aggregable,SegAVL}}
\geometry{left=20mm,right=20mm,top=20mm,bottom=20mm}
\parindent=0mm
\parskip=0mm

\begin{document}
\begin{center}
\pagenumbering{arabic}
{\huge \textsc{Segment Trees Haskell Implementation}}\\
\vspace{10mm} {\large Jiří Škrobánek\footnote[1]{Faculty of Mathematics and Physics, Charles University, {\ttfamily jiri@skrobanek.cz}}}\\
\vspace{10mm}\today, Prague

\end{center}

\section*{Introduction}
This document serves as a reference for \texttt{SegmentTree} Haskell module. The module provides an implementation of segment trees usable with any type of keys which implements \texttt{Ord} and can aggregate any associative function of values.

All editing operations take $\mathcal{O}(\log n)$ time on a tree with \textit{n} elements as well as range queries. Insertion and deletion is implemented using AVL-tree balancing algorithms yielding the same $\mathcal{O}(\log n)$ time complexity.

\section*{Example Usage}

In this example we build a tree on pairs of integers, to do this we must first instantiate the aggregation, in this case aggregation is simply addition. 

Firstly we build a tree on integers from 1 to 10. We update the value at 5 by adding 2 to it, lastly we ask for the sum of interval 2 to 8.

\lstinputlisting{../example.hs}

\section*{Complete List of Functionality}
\begin{lstlisting}
class Aggregable v where
aggregate :: (Maybe v) -> (Maybe v) -> (Maybe v)
\end{lstlisting}
Values must be an instance of this class.
\begin{lstlisting}
data SegAVL k v
| Node {key::k, value::v, lsub::SegAVL k v, rsub::SegAVL k v, height::Int, agg::Maybe v}
| Empty 
\end{lstlisting}
\subsection*{Functions}
\begin{itemize}
\item \begin{lstlisting}
segAVLBuildFromList :: (Ord k, Aggregable v) => [(k,v)] -> SegAVL k v
\end{lstlisting}
This function creates the segment tree from a sorted list of key-value pairs.
\item \begin{lstlisting}
segAVLInsertAndBalance :: (Ord k, Aggregable v) => SegAVL k v->k->v-> SegAVL k v
\end{lstlisting}
This function inserts the key-value pait into the tree, overwriting old values.
\item \begin{lstlisting}
segAVLSetValue :: (Ord k, Aggregable v) => SegAVL k v->k->v->(v->v->v)->SegAVL k v
\end{lstlisting}
Similar to the previous function, but takes an extra argument which handles the replacing of old value.
\item \begin{lstlisting}
segAVLDelete :: (Ord k, Aggregable v) => SegAVL k v -> k -> SegAVL k v
\end{lstlisting}
This function removes the given key from the tree.
\item \begin{lstlisting}
segAVLRange :: (Ord k, Aggregable v) => SegAVL k v -> k -> k -> Maybe v
\end{lstlisting}
This function return the aggregated value from interval between the second and third parameter.
\item \begin{lstlisting}
segAVLFind :: (Ord k, Aggregable v) => SegAVL k v -> k -> Maybe v
\end{lstlisting}
Returns the value associated with given key.
\item \begin{lstlisting}
segAVLMember :: (Ord k, Aggregable v) => SegAVL k v -> k -> Bool
\end{lstlisting}
Return a Boolean determining wheter the key is in the tree.
\item \begin{lstlisting}
segAVLToList :: (Ord k, Aggregable v) => SegAVL k v -> [(k,v)]
\end{lstlisting}
Produces an ordered set of key-value pairs from the tree.
\item \begin{lstlisting}
segAVLGetMin :: SegAVL k v -> Maybe (k,v)
\end{lstlisting}
Returns the minimum key-value pair present in the tree.
\item \begin{lstlisting}
segAVLGetMax :: SegAVL k v -> Maybe (k,v)
\end{lstlisting}
Returns the maximum key-value pair present in the tree.
\item \begin{lstlisting}
segAVLLowerBound :: (Ord k) => SegAVL k v -> k -> Maybe (k,v)
\end{lstlisting}
Return the key-value pair where the key is the smallest not lesser then the key provided.
\item \begin{lstlisting}
segAVLUpperBound :: (Ord k) => SegAVL k v -> k -> Maybe (k,v)
\end{lstlisting}
Return the key-value pair where the key is the greatest lesser then the key provided.
\end{itemize}
\end{document}